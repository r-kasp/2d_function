\documentclass[a4paper,12pt]{article}
\usepackage[T2A]{fontenc}
\usepackage[utf8]{inputenc}
\usepackage[english,russian]{babel}
\usepackage{amsmath}

%%%%%%%%%%%%%%%%%%%%%%%%%%%%%%%%%%%%%%%%%%%%%%%%%%%%%%%%%%%%

\begin{document}

Отчет:

Область параллелограмм с вырезом в правом нижнем угле

...A....................F
 
../..................../

./...........D....E
 
B.........C
 
Стороны AF и BC параллельны Оси X
 
В файле config.txt задаётся область:


 Задание области - параллелограмма

 Вырез в правом нижнем углу

 Верхний левый угол (x,y) - точка А

-0.5 0.5

 Нижний левый угол - точка B

-1 -0.5

 Верхний правый угол - точка F

1 0.5

 Верхний левый угол выреза - точка D

0.3 -0.3


остальные точки определяются однозначно
%%%%%%%%%%%%%%%%%%%%%%%%%%%%%%%%%%%%%%%%%%%%%%%%%%%%%%%%%%%%

\end{document}
